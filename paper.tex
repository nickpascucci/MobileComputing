\documentclass[a4paper,10pt]{article}
\usepackage[utf8x]{inputenc}
\usepackage[version=3]{mhchem}
\usepackage{fullpage}
\headsep = 25pt
\pagestyle{myheadings}
\markright{Nick Pascucci \hfill Motes \hfill}

\usepackage{color,hyperref}
\definecolor{darkblue}{rgb}{0.0,0.0,0.3}
\hypersetup{colorlinks,breaklinks,
            linkcolor=darkblue,urlcolor=darkblue,
            anchorcolor=darkblue,citecolor=darkblue}

% Uses hyperref to link DOI
\newcommand\doilink[1]{\href{http://dx.doi.org/#1}{#1}}
\newcommand\doi[1]{doi:\doilink{#1}}

% For \url{SOME_URL}, links SOME_URL to the url SOME_URL
\providecommand*\url[1]{\href{#1}{#1}}
% Same as above, but pretty-prints SOME_URL in teletype fixed-width font
\renewcommand*\url[1]{\href{#1}{\texttt{#1}}}

% Title Page
\title{Arduino-based Motes with XBee Radios}
\author{Nick Pascucci}
\date{09/27/2011}
\begin{document}
\maketitle

\section{Overview}
Ubiquitous computing relies heavily on sensor data. Having a variety of useful
devices located throughout an environment provides the system with the ability
to make better decisions. In general, these devices must be small and low power,
they must be wireless, and they must be cheap. Current technology allows us to
come closer to meeting all of these goals than ever before. Unfortunately, the
tools needed to develop applications against these systems are not widely
available or accessible.

On the other hand, the Arduino project has created an incredibly accessible and
easy to use platform for developing embedded computing projects. The Arduino
system combines a set of common development boards with a simple development
environment and a set of C++ tools which make embedded systems accessible. There
are a huge number of libraries and projects which make use of the Arduino or
provide additional functionality to the platform. I decided that this
characteristic made the Arduino a good starting point for an accessible sensor
node platform.

This paper will describe the overall architecture of the system. I will discuss
both the producer and consumer aspects of the sensor system, illustrating the
challenges I encountered while implementing it and the solutions I used to
resolve them.

\section{Producers}
Sensors are where it all begins. The data collection in my system is performed
by a pair of Arduino microcontrollers, which can easily be interfaced to a
variety of different sensors. For the purposes of the system's development, I
ran a simple program (see xbee.pde) which outputs an array of 20 bytes every
second to the attached radio. This firmware could easily be augmented and
extended to read data from a photoresistor, IR motion detector, laser tripwire,
I2C temperature sensor, or any of a variety of other devices. Data would be
buffered onboard the sensor and transmitted when 20 bytes had been collected.

Transmission occurs using a Digi XBee Series 2 2.4GHz radio. These radios are
programmable ZigBee compliant devices which operate in the 2.4GHz unlicensed
band at a transmit power of 50mW. The radios I used had chip antennas, which
improves their overall size at the expense of operating range. The devices are
easily used as part of a point-to-point or point-to-multipoint mesh
network. This is facilitated by their extensive AT command set and support for
both remote and local API modes. 

Fortunately, other developers have recognized the potential uses for these
radios as well. I made use of a preexisting XBee library (available at
http://code.google.com/p/xbee-arduino/) to handle interactions with the
device. The library abstracted the need to construct the packets and AT commands
which instruct the radio to transmit, and allowed me to proceed much more
quickly. The library also proved invaluable in deciphering the XBee API2
protocol, which was necessary on the consumer side. 

Data collected by the sensors is broadcast to the ``harvester'' node, which
consists of a single XBee radio attached to a desktop computer via a USB to
serial converter. The converter allows the host machine to see the XBee as a
serial port to which all of the input and output is directed.

\section{Consumer}
Data produced by the sensor nodes is tunneled to a Python program running on the
host computer via the serial port. This program monitors the serial port and a
server socket, opening new connections as clients connect and forwarding the
received data to them as it becomes available. It also performs a translation
between the XBee protocol into an application-level protocol mirroring that used
by CrossBow electronics' mote application. (This is the same protocol used in our
in-class assignment.)

In order to remain responsive to the serial port, server socket, and each
client, the program must be multi-threaded. This posed a difficult architectural
problem, as the ideal structure for the system was not immediately obvious. My
first inclination was to design the system such that a new thread was created
for each client from the main thread, which would contain the server
socket. Another thread would monitor the serial port. This required a large
amount of shared state, and necessitated communication mechanisms not well
supported by the existing libraries to relay new serial data to the client
threads.

I eventually decided that this design was untenable and sought to simplify
it. The design I settled on used only four threads: The main thread, one server,
one client, and one serial monitoring thread. The server thread waits for
incoming connections and enqueues them into a thread-safe FIFO queue. This queue
is periodically emptied by the client thread, which adds each client socket into
an internal list. The serial thread monitors the serial port, and adds data to a
second queue as it becomes avaiable (generally, one byte at a time). The client
thread reads this new data and passes it to a dedicated packet builder which
assembles the data one byte at a time into a packet. When a full packet has been
received, the client thread parses it into the proper form for the chosen
protocol, and transmits it to each client. The main thread simply waits for the
user to press the Enter key and terminates the program when this occurs.

\section{Extension}
The system is designed to be very loosely coupled. The sensors themselves are
able to communicate with any ZigBee compliant device, and use standard protocols
to communicate with each other. This allows new sensors from different
manufacturers to be added to the network as needed. As serial protocols are well
standardized, the link between the network and the host computer is also of
little concern. Any serial streaming device can be used to send data to the
machine; this could be anything from one microcontroller to millions of nodes in
a distributed network to a stand-in software device emulating a serial port. The
parsing and packetization of the data is performed in two files which are
separate from the main client-handling logic, enabling them to be easily
replaced by new handlers (Strategy pattern). Clients wishing to receive the data
can simply connect over standard TCP to the monitoring program and receive data
in a format dictated by the aforementioned files.

This loose coupling allows new applications to be developed easily, while
reusing as much of the existing structure as possible. Certain aspects of the
program such as client handling are highly unlikely to change; while others,
such as parsing and packetization, are very application dependent. These two
categories are kept well separated and are meant to encourage developers to
adapt the system to their needs.

\section{The Future}
Now that we have a platform that can be easily modified and supports open
end to end development, what applications are there? We have already discussed
this in generalities; what is needed now is a practical discussion. 

The benefit of these systems can be realized in a variety of ways. One the comes
to mind first is simply the distributed monitoring of environmental
conditions. Attaching a set of environmental sensors, such as humidity,
pressure, lighting, and temperature sensors, one can use the system to monitor
anything from the interior of a building to a large outdoor area. This data
could be used to adjust the building HVAC systems, or to monitor the healthiness
of an ecosystem. This sort of monitoring has numerous scientific and commercial
applications.

A second potential use could be the collection of usage information in a
commercial structure. Based on the utilization patterns of various areas and
features, a museum, shop, stadium, or any number of other locations could be
evaluated for efficient use of resources such as exhibits or vendors. This could
be accomplished rather simply on a set of networked Arduinos. 

Finally, I propose that the system could also be applied towards intelligent
manufacturing. The system could be used to monitor usage of spare parts or
consumables in a repair shop, factory, or storefront and order new components to
be manufactured as needed.

There are certain to be more applications which will be made apparent in the
future. This system, though simple, is extensible enough that it should be able
to meet their needs for the foreseeable future.


\vfill
\textit{This work is licensed under the Creative Commons Attribution-ShareAlike
  license. You are free to share, remix, and make commercial use of this work
  under the condition that you provide attribution to the author and share all
  derivative works under this license. Read the full text at the
  \href{http://creativecommons.org/licenses/by-sa/3.0/}{Creative Commons Website}.}
\end{document}          
